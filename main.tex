\documentclass[12pt]{report}
\usepackage[utf8]{inputenc}
\usepackage{datetime}

% graphicx
\usepackage{graphicx}
\graphicspath{{images/}}

% biblatex
\usepackage[square,numbers]{natbib}
\bibliographystyle{unsrtnat}

\usepackage[nottoc]{tocbibind}

\title{
    {Deep Q-learning in Grid Worlds}
}
\author{Mihnea Ungureanu}

\begin{document}

\maketitle

\chapter*{Abstract}
Over the relatively recent years, \textbf{reinforcement learning (RL)} has gained immense popularity.
A multitude of interesting results in this field focuses on agents learning in simple, game-like, 2D grid-based environments.

In this paper, we focus on one particular technique in RL -- Q-learning -- and its applications in grid-based, game-like environments.

Classic Q-learning was an early breaktrhough in RL.
We document its evolution and build a survey of improvements over the original.
These techniques are used in practice and have good results in certain areas, such as playing Atari games.
We give examples of succesful applications and explain why Q-learning played a key role.
Finally, we look at advantages and disadvantages over other techniques in RL.

\tableofcontents

\chapter{Short Introduction to RL}
\section{What is Reinforcement Learning?}

\textbf{Reinforcement learning (RL)} is a subset of machine learning that has its origins in computer science and artificial intelligence techniques established in the 1950s.
RL is unique within machine learning as it is focused on goal-directed learning from agent-environment interaction\cite{rlai}.
Simply put, agents learn from environment interactions and gain experience.
They use this experience to optimize their behaviour to achieve problem-defined goals.
This nature-inspired approach captures a broader definition of intelligence (in a human sense) than systems that reason about the world using a finite set of logical rules (i.e., knowledge-based systems).
Like most other subfields of ML, RL is a class of problems as well as a class of solutions to those problems\cite{rlai}.

Advances in the field of deep learning (a broad family of methods based on artificial neural networks\cite{wiki:Deep_learning}) that could enhance the classical algorithms used in RL led to the expansion of the field -- marking the birth of \textbf{deep reinforcement learning (deep RL)}.
Deep RL has surged recently after a series of new algorithms and succesful applications were released. In the following paragraphs, we will summarize some of the most significant results.

\textbf{AlphaGo} (2015)\cite{ago} managed superhuman performance in the game of Go, beating 18-time world champion Lee Sedol.
In late 2017, the AlphaGo family was expanded to contain AlphaGo Zero and AlphaZero\cite{azero}.
Both systems have managed to surpass all their predecessors in game performance, as well as training efficiency \cite{wiki:AlphaGo} (a lot less training time was required to beat the record).
The key difference was that their training was done with absolutely no expert knowledge for either system (hence the ``zero'').
Instead, everything was learned by self-play \cite{azero}.

\textit{Mnih et al.}'s \textbf{DQN}\cite{atari-dqn} (2013) managed human-level performance at seven classic Atari games.
Following the success of DQN, the community worked to improve the standard algorithms.
\textbf{Rainbow DQN} (2017)\cite{rainbow-dqn} is the culmination of that work.
The authors succesfully combined multiple high-quality improvement methods over the original DQN.
The resulting algorithm set a new record on the Atari benchmark.

\textbf{OpenAI Five}\cite{openai-dota} (2018) produced a bot that managed to beat top professional Dota 2 players in international championships.

Despite most of the above list being focused on achievements in video games, RL has been applied to other problems including robot control, elevator scheduling, telecommunications\cite{wiki:Reinforcement_learning}.

\section{Fundamental Components}

First of all, we present the agent-environment feedback loop, as it is crucial to understand the dynamic. An agent acts upon the environment, which reacts according to its set of governing rules.
\begin{enumerate}
    \item The agent receives a reward for this action. This (immediate) reward signal is problem-defined and quantifies the agent’s goal. (see the Reward Hypothesis).
    \item The environment reaction sends the agent into the next state.
    \item The agent perceives and needs to decide on its next action.
\end{enumerate}

\begin{figure}[h]
    \caption{The interaction loop. (Partial reproduction from \cite{rlai}, 3.2)}
    \centering
    \vspace*{0.5cm}
    \includegraphics[width=0.5\textwidth]{agent-env-fig}
\end{figure}

\subsection{State and Policy}

We use the concept of \textbf{state}. The state most often refers to the internal state of the agent. Although, in some contexts, we can use it to mean the environment state.
A state contains relevant information from the environment at a given time-step.
A key point in RL is that a well-engineered (agent) state captures all the relevant information and removes the need of explicitly memorizing state history.
This is mathematically by described the \textbf{Markov property}\cite{silver-lectures}, a fundamental property of the mathematical framework underpinning RL -- Markov decision processes (MDPs).

A \textbf{policy} completely characterizes the agent’s behaviour.
``It is a mapping from perceived states of the environment to actions to be taken in those states'' \cite{rlai}.
Policies can be deterministic (i.e. ``if this then that'' rules) or stochastic.
A \textbf{stochastic policy}\cite{silver-lectures} is a probability distribution over actions, given a state.

\subsection{Reward}
``The reward signal is the primary basis of altering the policy.'' \cite{rlai}.
The \textbf{reward} models the problem-defined goals as a scalar that can be associated with each state transition.
The assertion that we can completely and correctly model all goals using reward functions is central to the field of RL.
This is called the \textbf{Reward Hypothesis} and is formulated below:
\begin{quotation}
    That all of what we mean by goals and purposes can be well thought of as maximization of the expected value of the cumulative sum of a received scalar signal (called reward). \textit{(from \cite{rlai}, Chapter 3.2)}
\end{quotation}

The \textbf{return} denotes the cumulative reward obtained by the agent over time.
This is what a RL system is supposed to maximize.
There are multiple mathematical models used to represent the return.
The simplest example is to simply sum the reward over a finite number of steps.
However, in practice we often use \textbf{discounting}.
Discounting is, simply, a way to control how much the agent cares about future rewards.
More on this topic can be found in either theorethical reference \cite{rlai,silver-lectures}.
% In the actual paper, elaborate here.

\subsection{Value Function}\label{rl:value-func}

A \textbf{value function} measures how good each state is, with regard to the long-term potential of that state.
``Whereas rewards define the immediate desirability of an environmental state'', a value function ``indicates the long-term desirability of a state'' \cite{rlai}.
The value of a state is given by two things:
\begin{enumerate}
    \item the immediate value of being in that state (that state’s immediate reward)
    \item the potential return going forward in time from that state.
\end{enumerate}

This models a \textbf{long-term thinking aspect} into learning, as choosing one state over another often excludes future paths of action.
Let us take a game of chess as example.
If one move leads to the opponent capturing one of the agent's pieces, no future actions can be executed with that piece.
Thus, making a move with a high initial reward (for example, the agent trades a pawn for a knight), excludes -- for example -- the option of \textbf{promoting} that particular pawn (a later, higher reward).

\subsection{Model}
A RL system may or may not have a \textbf{model}.
A \textbf{model} (of the environment) allows the agent to plan and make predictions of its environment.

Some algorithms focus explicitly on learning a model and use it for \emph{planning}.
This approach is called \textbf{model-based}.
In this case, an agent can query the model to simulate what would happen with the environment, before actually choosing an action (hence the planning aspect).

Approaches without a model are called \textbf{model-free}.
Model-free agents are ``explicitly trial-and-error learners'' \cite{rlai}.

\begin{figure}[ht]
    \caption{A way of classifying RL methods based on whether it has a value function, policy or model. (Reproduced from David Silver's lectures. \cite{silver-lectures})}
    \vspace*{0.2cm}
    \centering
    \includegraphics[width=0.65\textwidth]{silver-venn}
\end{figure}


\chapter{Q-learning and Deep Q-networks}
\section{Naive Q-learning}

\section{Deep Q-networks}

\section{Atari DQN}

\section{Rainbow DQN: The state-of-the-art}

\chapter{Application in Grid Worlds and 2D Games}
\section{Playing Atari Games}

The aforementioned papers all use the Atari 2600 environment to train and test agent implementations.

We remark in the introduction chapter that \emph{original DQN} \cite{atari-dqn} study produced agents able to surpass human-level performance in \emph{Breakout}, \emph{Enduro} and \emph{Pong}.
The subsequent improvements to the original have of course managed to equal or surpass the original implementations \cite{ddqn-paper,per-paper}.

The \emph{Atari 2600 Learning Environment (ALE)} \cite{ale-paper} has become a popular benchmark used in the study of AI.
One of the reasons being that Atari games are often intuitive and fun to watch.
Researchers can often have a good measure of how well an agent behaves only by observing its performance on the Atari emulator.

\begin{figure}[]
    \includegraphics[width=\textwidth]{atari-screenshots}
    \centering
    \caption{Screen captures from different games on the Atari 2600 (Corresponds to fig. 1 in \cite{atari-dqn})}
    \label{fig:atari-games}
\end{figure}

\section{Roguelike Games}

\emph{Rogue} is a video game originally developed around 1980 for Unix-based mainframe system \cite{wiki:Rogue_(video_game)}.

\begin{figure}[]
    \includegraphics[width=0.75\textwidth]{rogue}
    \centering
    \caption{The iconic ASCII-based interface of rogue (from \cite{wiki:Rogue_(video_game)})}
    \label{fig:rogue}
\end{figure}

As a staple of the \emph{dungeon crawling} genre, the game requires the player to navigate \emph{dungeons}.
The layout of a dungeon is visible -- it consists of several rooms.
The player may start out in any one of them.
The exact position of every item, monster or door is unknown.
The game features exploration mechanics like searching for keys to open locked doors and hidden passages.

\emph{Rogue} would not be complete without its \emph{combat aspect}.
Although rudimentary, it was a fun, innovative mechanic for its time.
The game features monster encounters.
Monsters are visible to the player once they are close enough.
Their initial positions on a map are chosen at random and they have resistances that one has to find a way around and vulnerabilities that one can exploit.

An interesting feature, which distinguishes \emph{Rogue} from other adventure games of the era, is \emph{random generation}.
The game builds randomly-generated dungeon layouts and randomizes the placement of items and monsters inside it \cite{wiki:Rogue_(video_game)}.

Insipired by the simple yet fun qualities of this game, the genre of \emph{roguelikes} was born.
Part of this genre are games that enhance the original experience prodiving new mechanics or improving the original ones.
It is essentially a mix-match of the properties of the original game: random generation, one-life, enemy combat and item system.

We consider roguelikes to be an appropriate level of complexity for RL agents to handle and still thrive.
Rogue has the potential of creating interesting behaviour patterns which would help find and address problems in RL (and any specific method applied i.e. Q-learning).
One of the problems of RL is \textbf{generalization}:
an agent having developed a generalized skill is more useful than one which has simply managed to memorize its environment.
Rogue's randomly generated levels provide a good environment to test this.

There are studies using \emph{roguelikes} as an environment to study RL agents.
These studies aimed to train agents that produced optimal behaviour in environments closely matching the original game.
Below, we are going to present one case of significance.

In the \emph{Rogue-Gym} paper \cite{rogue-gym-paper}, researchers propose using Rogue as a benchmark for generalization.
As we mentioned above, one of Rogue's characteristics is random level generation.
This can be leveraged to study whether an agent has learned a skill or has just learned to retrace a certain environment.
In this paper, researchers use state-of-the-art policy optimization methods to train agents on a variant of Rogue without combat.
Its results are pointing out that most tested methods have poor generalization score.
This has to do both with Rogue's complexity as well as the shortcomings of the RL algorithms.

\chapter{Conclusion}
The transfer of knowledge from deep learning to the classical methods in RL is revolutionizing the field.
Q-learning has evolved a long way since its inception in 1992.
Off-policy TD-control methods have shown a promising evolution, shifting from inefficient tabular representations to neural networks and other powerful ML models.

At the same time, despite being marked by the success of AlphaGo and its variants, RL agents are not yet prepared to handle real-world problems.

The study of autonomously learning RL agents in games has also been challenging.
As we have seen with DQN and its successors, its greatest weakness remains sample efficiency -- agents need to train for millions of steps before they demonstrate any significant amount of ability.
And even then -- as pay closer attention to generalization in deep RL -- we find that most methods fail to manifest the ability to generalize, outside of very simple strategies (such as Breakout or Pong).

With this in mind, the future looks bright, with a new recently released paper from DeepMind -- \emph{Agent57} (2020) \cite{agent57-paper}.
The latest research trends seem to be using mixed approaches: DQN and policy optimization algorithms, in an attempt to break the barriers set by their predecessors.
Agent57 has managed to break every record on the Atari benchmark and has obtained scores above the human baseline on each of the 57 games in the set.


\bibliography{references}

\end{document}